% !TEX program = pdflatex
\input{preamble/preamble.tex}
\input{preamble/preamble_math.tex}

\usepackage{macros}

% \linenumbers

\title{Latent Hawkes Processes}
\author{Scott W Linderman}
\date{November 1, 2016}

\begin{document}
\maketitle

\section{Model}
Consider a scenario in which the data we observe is driven by
a set of hidden events,
\begin{align}
  \mcH &= \{s_j, m_j, \theta_j\}_{j=1}^J,
\end{align}
where~$s_j \in [0,T]$ is the time of the~$j$-th hidden
event,~${m_j \in \{1, \ldots, M\}}$ is the latent node on which it
occurred, and~$\theta_j$ is a set of properties associated with that
event. For example, the latent nodes could be unobserved neurons in
the brain, the timestamps could correspond to the spikes fired by
those neurons, and the parameters could correspond to some relevant
properties of those spikes. Alternatively, perhaps there is only one
latent node ($M=1$) which roughly captures things that happen in the
outside world, where the timestamps reflect the timestamps of those
real-world events and the parameters represent properties of the
event.

We will assume that these hidden events are sampled from a
marked point process with independent rates,~$\lambda_m(s,\theta)$
for each latent node~$m$. The set~$\mcH$ is then the union of
events from all latent nodes. In the simplest case, we will assume
that,
\begin{align}
  \lambda_m(s, \theta) = b_m \cdot p(\theta),
\end{align}
where~$b_m$ is a homogeneous background rate and~$p(\theta)$ is
the density of a prior distribution over properties~$\theta$. 

The observed data is a set of events on~$N$ visible nodes.
In this model, the~$j$-th latent event gives rise to a set of child
events, which we denote by,
\begin{align}
  \mcD_j &= \{(t_k, n_k, x_k)\}_{k=1}^{K_j},
\end{align}
where~$K_j$ is the number of observed events induced by the latent
event~$j$,~$t_k \in (s_j,T]$ is the time of the induced event (note
that the system is causal), and~$x_k$ is the content of the observed
event. We will further refine this notation by defining,
\begin{align}
  \mcD_{j,n} &= \{(t_k, x_k): n_k = n\},
\end{align}
such that~$\mcD_j = \cup_{n=1}^N \mcD_{j,n}$.
Furthermore, assume that there is a static background model
that gives rise to an additional set of events on each node,
\begin{align}
 \mcD_0 &= \{(t_k, n_k, x_k)\}_{k=1}^{K_0},
\end{align}
with~$\mcD_{0,n}$ defined analogously to~$\mcD_{j,n}$. With this
notation in place, we can now define the observed data as
the union of all of these sets,
\begin{align}
  \mcD &= \bigcup_{n=1}^N \mcD_{0,n} \bigcup_{j=1}^J \mcD_{j,n}.
\end{align}
Let~$K=K_0 + \sum_{j=1}^J K_j$ denote the total number of observed events.

To make this concrete, the observed events may be cables sent by a
collection of~$N$ embassies, and the mark,~$x_k \in \naturals^V$ may
be the word counts of the~$k$-th message (with vocabulary
size~$V$). The latent events correspond to things happening in the
outside world, which have an associated distribution over
words,~$\theta_j$.

In another setting,~$x_k \in \reals$ may be an associated event
amplitude, like the size of a post-synaptic potential in a recording
of a neuron's voltage. These events are induced by either the
background rate (noise) or by one of a set of hiddent events, like
spikes on pre-synaptic neurons.

\begin{figure}[t!]
\includegraphics[width=5in]{latent_hawkes_example.pdf}
\caption{An example of a draw from a latent Hawkes process
a single latent node ($M=1$) which gives emits the ~$J=25$ hidden
events denoted by black dots. Each latent event has a time
stamp,~$s_j$, from a homogeneous Poisson process and a
parameter~$\theta_j \sim \distNormal(0,1)$, marked by the height of
the dot. These latent events influence the conditional intensity of
the marked point processes for each of the~$N=3$ observed nodes. In
this case, there are different weights~$w_{1 \to n}$ for each of the
three node, but all have the same background rate,~$b_n=0.5$. The
observed events are either drawn from the background model with
standard Gaussian content,~$x_k$, or from an impulse response induced
by a latent event. If the latter, the time is drawn from~$t_k =
s_{z_k} + \mcE(1)$ and the content is drawn from
$x_k \sim \distNormal(\theta_{z_k}, 0.5^2)$. Our goal is to infer the
times and parameters of the black dots given observations of the blue
and red dots.}
\end{figure}

To relate the hidden and observed events,
we consider a model in which the observed datasets follow a marked
point process,
\begin{align}
  \mcD_{j,n} &\sim \mcP(\lambda_{j,n}(t, x)),
\end{align}
where~$\lambda_{j,n}(t, x): [0,T] \times \mcX \to \reals_+$ denotes
the conditional intensity function of events on node~$n$ induced by
the~$j$-th hidden event.  We further assume that this conditional
intensity function simplifies to,
\begin{align}
  \lambda_{j,n}(t, x)
  &= w_{m_j \to n} \, g(t-s_j; \theta_j) \, h(x; \theta_j) \, \bbI[t > s_j],
\end{align}
where $w_{m \to n} \in \reals_+$ is the weight of influence hidden
node~$m$ has on observed node~$n$; ${g(\Delta
t; \theta_j): \reals_+ \to \reals_+}$ is a normalized impulse response
function (i.e.~$\int_0^\infty g(\tau; \theta_j) \, \mathrm{d}\tau =
1$); and~$h(x; \theta_j)$ is a probability density on the space~$\mcX$
parameterized by~$\theta_j$.

Likewise, define the background intensity of node~$n$ as,
\begin{align}
  \lambda_{0,n}(t,x) &= b_n \, h(x; \theta_n),
\end{align}
where~$b_n$ is the time-homogeneous rate of events
and~$h(x; \theta_n)$ is a probability density over event content
specific to node~$n$ (permitting a slight overloading of the indices
for the background parameters,~$\theta_n$).

Since the observed
data is the union of sets of events drawn from independent marked point processes,
the total conditional intensity for node~$n$ is given by,
\begin{align}
  \lambda_n(t, x) &= b_n \, h(x; \theta_n) + \sum_{j=1}^J w_{m_j \to n} \, g(t-s_j; \theta_j) \, h(x; \theta_j) \, \bbI[t > s_j].
\end{align}
\TODO{This is the critical step -- I'm pretty sure that this
is right, but we should discuss.}
In other words, the events follow a marked Hawkes process.


\section{Inference}
The log likelihood is,
\begin{align}
  \log p(\mcD \given \mcH) &= -\sum_{n=1}^N \left[\int \lambda_n(t,x) \, \mathrm{d}t \, \mathrm{d}x \right] + \sum_{k=1}^K \log \lambda_{n_k}(t_k, x_k) \\
  &= -\sum_{n=1}^N \left[b_n T + \sum_{j=1}^J w_{m_j \to n} \right] + \sum_{k=1}^K \log \lambda_{n_k} (t_k, x_k),
  \label{eq:loglkhd}
\end{align}
ignoring edge effects in the integral. 

From the generative process described above, we see that there is an
additional unobserved variable, namely, the parent event~$j$ that gave
rise to the event~$k$.  To perform inference, we can reinstantiate
this latent variable. Let ~$z_k \in \{0, \ldots, J\}$ indicate which
of the additive components of the conditional intensity function gave
rise to the~$k$-th event. With these parent variables, the
``complete data'' log likelihood decomposes into a sum,
\begin{multline}
  \log p(\{(t_k, n_k, x_k, z_k)\}_{k=1}^K \given \mcH) =
  -\sum_{n=0}^N \left[ b_n T + \sum_{j=1}^J w_{m_j \to n} \right] \\
  + \sum_{k=1}^K \bbI[z_k=0] \log b_{n_k} 
  + \sum_{j=1}^J \bbI[z_k=j] \log \left(w_{m_j \to n_k} \, g(t_k-s_j; \theta_j) \, h(x_k; \theta_j) \, \bbI[t_k > s_j] \right)
\end{multline}

Given these parent variables, and given times of the latent
events~$\{s_j\}_{j=1}^J$, inference of the remaining parameters is
relatively straightforward. The weights and background rates have
conjugate gamma priors.

If~$\theta_j = (\tau_j, \eta_j)$, where~$\tau_j$
parameterizes the temporal density,~$g$, and~$\eta_j$ governs
the content density,~$h$, then in some cases we can choose conjugate
exponential family distributions. For the temporal kernel,
$g(\Delta t; \tau_j) = \mathrm{Exp}(\Delta t; \tau_j)$,
the likelihood is conjugate with a gamma prior on~$\tau_j$.
For Gaussian content, a normal inverse-gamma prior, and for
multinomial distributed content, a Dirichlet prior on~$\eta_j$,
are natural choices.

Moreover, the conditional distribution of the parent variables is given by,
\begin{align}
  z_k &\sim
  \begin{bmatrix}
        \pi_{k,0}, & \pi_{k,1}, & \ldots, &\pi_{k,J}
  \end{bmatrix}  \\
  \pi_{k,0} &\propto b_{n_k} h(x_k; \theta_{n_k}) \\
  \pi_{k,j} &\propto w_{m_j \to n_k} \, g(t_k - s_j; \theta_j) \, h(x_k ; \theta_j) \, \bbI[t_k > s_j]; \qquad \text{for }j>0.
\end{align}
These could be sampled in parallel, or we could sample them
serially, collapsing out the parameters~$\theta$ based on
the assignments of events~$k' \neq k$ (as long~$g$ and~$h$ are conjugate
with the prior on~$\theta$). This ties all of the observed
events together and forces a serial updating of the~$K$ events.
In practice, this may lead to slower mixing than a parallel
implementation of an ``un-collapsed'' Gibbs sampler. 

\subsection{Inferring the latent event times}
The more challenging problem is that of updating the
set of latent event times. If we hold the parent assignments
fixed, then we have hard constraints on the time of the
~$j$-th event --- it is limited by the earliest time of
the events it gives rise to. Moreover, we cannot remove
a parent event if any children belong to it. To circumvent
this issue, we propose to marginalize out the parent
variables and directly evaluate the log likelihood of a
candidate set of hidden events,~$\mcH$, using Eq.~\eqref{eq:loglkhd}.

We could design as simple MCMC algorithm with a set of
Metropolis-Hastings transition operators to jitter, add,
and remove latent events. In each case, we can evaluate
the marginal likelihood in Eq.~\eqref{eq:loglkhd} as
well as the prior,~$p(\mcH)$, which is also a point
process likelihood.

Alternatively, we could consider joint updates of the
latent event sequences. Imagine binning time into~$T$
windows of width one.  We can write the log likelihood
as a sum over these time bins:
\begin{align}
\log p(\mcD \given \mcH) &=
\sum_{i=1}^T \left(
  -\sum_{n=1}^N \sum_{j: s_j \in [i, i+1)} w_{m_j \to n} 
  + \sum_{k: t_k \in [i, i+1)} \log \lambda_{n_k}(t_k, x_k) 
\right) + c \\
&= \sum_{i=1}^T \log p\big(\mcD[i] \, \big| \, \mcH[1:i]\big) + c,
\end{align}
where~$c=-\sum_{n=1}^N b_nT$ is constant wrt~$\mcH$; $\mcD[i]$
denotes the data in the~$i$-th time bin, and~$\mcH[1:i]$ denotes
the set of hidden events up to and including the~$i$-th time bin.
If we assume that the prior factorizes in a similar way, then the
conditional distribution,~$p(\mcH \given \mcD)$ will also factor
in this causal way. This suggests that we could estimate the
conditional distribution of~$\mcH$ using sequential Monte
Carlo\footnote{Or some other smoothing algorithm for non-linear non-Gaussian
time series}. Moreover, we could incorporate the inference of~$\mcH$
into a Markov chain using particle MCMC. Our proposal distributions
would be interesting --- we would probably want something like,
\begin{align}
  q(\{s_j, m_j=m, \theta_j\}) &= \mcP(\nu(s, \theta)),
\end{align} 
where~$\nu$ is a proposal measure on the time interval~$[i, i+1)$ and
the sapce of parameters. That is, in our SMC algorithm, we propose
a set of latent events for the next time interval, evaluate the
proposal probability, and then evaluate the joint probability of
the data (in that time bin) given the history of events through time~$i+1$. 



\bibliographystyle{apa}
\bibliography{refs.bib}

\end{document}
